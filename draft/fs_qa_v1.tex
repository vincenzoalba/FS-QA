\documentclass[11pt]{iopart}
\usepackage[english]{babel}
\pdfoutput=1
%\usepackage{amsmath}
\usepackage{wasysym}
\usepackage{booktabs}
\usepackage{amssymb}
\usepackage{amsbsy}
\usepackage{verbatim}
\usepackage{graphicx}
\usepackage{epstopdf}
\usepackage{color}
\usepackage{sidecap}
\usepackage{bm}% bold math
\usepackage{tikz}
%\usepackage[backend=bibtex]{biblatex}

\usepackage[colorlinks,bookmarks=false,citecolor=blue,linkcolor=red,urlcolor=blue]{hyperref}

\newcommand*\circled[1]{\tikz[baseline=(char.base)]{
            \node[shape=circle,draw,inner sep=2pt] (char) {#1};}}


%\usepackage{cite}

%
% WARNING !!!!
% 
% iopart.cls definition of \tableofcontents overwrites the
% short title printed every page. 
% The following redefinition of \tableofcontents fixes the
% the problem. 

\catcode`@=11 % If we need private TeX macros
\renewcommand\tableofcontents{%
  \section*{\contentsname}%
  \@starttoc{toc}%
}
\catcode`@=12% '@' is no more a character
\newcommand{\bra}[1]{\langle\left.{#1}\right|}
\newcommand{\ket}[1]{\left|{#1}\right.\rangle}


\def\be{\begin{equation}}
\def\ee{\end{equation}}
\def\u{\uparrow}
\def\d{\downarrow}
\def\nm{\newmoon}
\def\fm{\fullmoon}
\def\T{\rule{0pt}{.6cm}}
\def\B{\rule[-.4cm]{0pt}{0pt}}

\begin{document}

\setlength{\parindent}{0pt}


\title{The Quench-Action method in integrable spin chains: A Monte Carlo 
approach}

\author{Authors}
%\address{$^1$ International School for Advanced Studies (SISSA),
%Via Bonomea 265, 34136, Trieste, Italy,
%INFN, Sezione di Trieste}


\date{\today}



%%%%%%%%%%%%%%%%%%%%%%%%%%%%%%%%%%%%%%%%%%%%%%%%%%%%%%%%%%%%%%%%%%%%%%%%%%
\begin{abstract} 


fdasfa
\end{abstract}

\maketitle

%%%%%%%%%%%%%%%%% INTRODUCTION %%%%%%%%%%%%%%%%%%%%%
\section{Introduction}
\label{intro}

We show that it is possible to numerically simulate the Quench Action approach 
combining Monte Carlo methods and Bethe ansatz techniques. 

We focus on the situation in which the pre-quench initial state is the Neel 
state or the Majumdar-Ghosh state. 

We investigate the importance of the zero-momentum strings in the Quench Action. 

Without zero-momentum strings the overlap saturation rules are not valid, 
i.e., in finite size systems the vast majority of the eigenstates contain 
zero momentum strings. 

The details on the eigenstates counting depend on the pre-quench initial state. 

However, we show that one can restrict to the set of non-zero momentum strings. 
The fact that one neglects zero-momentum strings gives rise only to scaling 
corrections. 

We also investigate the validity of the Bethe-Takahashi approximation for the 
calculation of the overlap. 

In the thermodynamic limit it is natural to assume that the steady-state expectation 
values are obtained using the so-called diagonal ensemble, which is defined as 
%
\begin{equation}
\label{d-ensemble}
\langle{\mathcal O}\rangle=\sum\limits_{\alpha}|\langle\Psi_0|\alpha\rangle|^2
\langle\alpha|{\mathcal O}|\alpha\rangle. 
\end{equation}
%


%%%%%%%%%%%%%%%%%%%%%%% BETHE ANSATZ APPROACH %%%%%%%%%%%%%%%%%%%%%%%%%%%%%%
\section{The Heisenberg spin chain}
\label{xxx-chain}

Here we consdier the spin-$\frac{1}{2}$ isotropic Heisenberg chain ($XXX$ chain). 
The $XXX$ chain with $L$ sites is defined by the Hamiltonian 
%
\begin{equation}
\label{xxx-ham}
{\mathcal H}\equiv J\sum\limits_{i=1}^L\left[\frac{1}{2}(S_i^+S^-_{i+1} 
+S_i^{-}S_{i+1}^+)+S_i^zS_{i+1}^z\right],  
\end{equation}
%
where $S^{\pm}_i\equiv (\sigma_i^x\pm i\sigma_i^y)/2$ are spin operators acting on the 
site $i$, $S_i^z\equiv\sigma_i^z/2$, and $\sigma^{x,y,z}_i$ the Pauli matrices. We fix 
$J=1$ and use periodic boundary conditions, identifying sites $L+1$ and $1$. The total 
magnetization $S_{T}^z\equiv\sum_iS_i^z=L/2-M$, with $M$ number of down spins (particles), 
commutes with~\eref{xxx-ham}, and it is here used to label its eigenstates. 


%%%%%%%%%%%%%%%%%%%%%%%%%%%%%%%%%%%%%%%%%%%%%%%%%%%%%%%%%%%%%%%%%%%%%%%%%%%
\subsection{Bethe equations and wavefunctions}
\label{bethe_equations}

The generic eigenstate of~\eref{xxx-ham} in the sector with $M$ particles can be written as 
%
\begin{equation}
\label{ba_eig}
|\Psi_M\rangle=\sum\limits_{1\le x_1<x_2<\dots<x_M\le L}A_M(x_1,x_2,
\dots,x_M)|x_1,x_2,\dots,x_M\rangle,
\end{equation}
%
where the sum is over the positions $\{x_i\}$ of the particles, and $A_M(x_1,
x_2,\dots,x_M)$ is the eigenstate amplitude corresponding to particles 
at positions $x_1,x_2,\dots, x_M$. $A_M(x_1,x_2,\dots, x_M)$ is given as 
%
\begin{equation}
\label{ba_amp}
A_M(x_1,x_2,\dots,x_M)\equiv\sum\limits_{{\mathcal P}\in S_M}\exp\Big[i
\sum\limits_{j=1}^Mk_{{\mathcal P}_j}x_j+i\sum\limits_{i<j}\theta_{{
\mathcal P}_i{\mathcal P}_j}\Big].
\end{equation}
%
Here the outermost sum is over the permutations $S_M$ 
of the so-called quasi-momenta $\{k_1,k_2,\dots,k_M\}$. The two-particle 
scattering phases $\theta_{m,n}$ are defined as 
%
\begin{equation}
\label{s_phases}
\theta_{m,n}\equiv \frac{1}{2i}\log\Big[-\frac{e^{ik_m+ik_n}-2e^{ik_m}+1}
{e^{ik_m+ik_n}-2e^{ik_n}+1}\Big].
\end{equation}
%
The energy associated to the eigenstate~\eref{ba_eig} is  
%
\begin{equation}
\label{ba_ener}
E=\sum\limits_{\alpha=1}^M(\cos(k_\alpha)-1).
\end{equation}
%
The quasi-momenta $\{k_\alpha\}$ are obtained by solving the so-called 
Bethe equations  
%
\begin{equation}
\label{ba_eq}
e^{ik_\alpha L}=\prod\limits^M_{\beta\ne\alpha}\Big[-\frac{1-2e^{
ik_\alpha}-e^{ik_\alpha+ik_\beta}}{1-2e^{ik_\beta}-e^{ik_\alpha+
ik_\beta}}\Big].
\end{equation}
%
It is useful to  introduce the rapidities $\{\lambda_\alpha\}$ as 
%
\begin{equation}
\label{rap}
k_\alpha=\pi-2\arctan(\lambda_\alpha)\quad\mbox{mod}\, 2\pi.
\end{equation}
%
Taking the logarithm on both sides in~\eref{ba_eq}, and using~\eref{rap}, 
one obtains the Bethe equations in logarithmic form as 
%
\begin{equation}
\label{ba_eq_log}
\arctan(\lambda_\alpha)=\frac{\pi}{L}J_\alpha+\frac{1}{L}\sum\limits_{
\beta\ne\alpha}\arctan\Big(\frac{\lambda_\alpha-\lambda_\beta}{2}\Big),
\end{equation}
%
where $-L/2<J_\alpha\le L/2$ are the so-called Bethe quantum numbers. The 
$J_\alpha$ are half-integers and integers for $L-M$ even and odd. 


These solutions of the Bethe equations~\eref{ba_eq} form particular ``string'' patterns 
in the complex plane, in the limit of large chains $L\to\infty$(string hypothesis)~\cite{
bethe-1931,taka-book}. Specifically, rapidities forming a ``string'' of length $1\le 
n\le M$ (that we defined here as $n$-string) are parametrized as 
%
\begin{equation}
\label{str_hyp}
\lambda^{j}_{n;\gamma}=\lambda_{n;\gamma}-i(n-1-2j)+i\delta_{n;\gamma}^j,\qquad 
j=0,1,\dots, n-1, 
\end{equation}
%
where $\lambda_{n;\gamma}$ is the real part of the string (string center), 
and $\gamma$ labels strings with different centers, while $j$ labels the different 
components of the string. In~\eref{str_hyp} $\delta_{n;\gamma}^j$ are the string 
deviations, which typically vanish exponentially in the thermodynamic limit. 

Notice that pure real rapidities are strings of unit length ($1$-strings). 


%%%%%%%%%%%%%%%%%%%%%%%%%%%%%%%%%%%%%%%%%%%%%%%%%%%%%%%%%%%%%%%%%%%%%%%%%%%
\subsection{Bethe-Gaudin-Takahashi (BGT) equations} 

The string centers $\lambda_{n;\gamma}$ in~\eref{str_hyp} are obtained by solving the 
so-called Bethe-Takahashi equations
%
\begin{equation}
\label{bt_eq}
2L\theta_n(\lambda_{n;\gamma})=2\pi I_{n;\gamma}+\sum\limits_{(m,
\beta)\ne(n,\gamma)}\Theta_{m,n}(\lambda_{n;\gamma}-\lambda_{m;\beta}), 
\end{equation}
%
where the generalized scattering phases $\Theta_{m,n}$ read 
%
\begin{eqnarray}
\nonumber\fl\Theta_{m,n}(x)\equiv\left\{\begin{array}{cc}
\theta_{|n-m|}(x)+\!\!\!\!\!\sum
\limits_{r=1}^{(n+m-|n-m|-1)/2}\!\!\!\!\!2\theta_{|n-m|+2r}(x)
+\theta_{n+m}(x) & \quad\mbox{if}
\quad n\ne m\\\fl\sum\limits_{r=1}^{n-1}2\theta_{2r}(x)+
\theta_{2n}(x) & \quad\mbox{if}\quad n=m
\end{array}\right.
\end{eqnarray}
%
and $\theta_\alpha(x)\equiv 2\arctan(x/\alpha)$. Here $I_{n;\gamma}$ are the 
Bethe-Takahashi quantum numbers associated with $\lambda_{n;\gamma}$. 

Each $M$-particle eigenstate can be characterized by its ``string content'' ${
\mathcal S}\equiv\{s_1,\dots,s_M\}$, with $s_n$ the number of $n$-strings. 

It can be shown that $I_{n;\gamma}$  are integers or half-integers for $L-s_n$ 
odd and even, respectively. 


Clearly, the constraint $\sum_{\alpha=1}^{M}\alpha s_\alpha=M$ 
has to be satisfied. The upper bound for the Bethe-Takahashi quantum numbers 
can be derived as   
%
\begin{equation}
|I_{n;\gamma}|\le I^{(MAX)}_{n}\equiv\frac{1}{2}(L-1-\sum
\limits_{m=1}^Mt_{m,n}s_m),
\label{bt_qn_bound}
\end{equation}
%
where $t_{m,n}\equiv 2\mbox{min}(n,m)-\delta_{m,n}$. 

In the thermodynamic limit the BGT equations become a set of coupled integral 
equations given as 
%
\begin{equation}
\label{bgt-th}
a_n(\lambda)=\rho_n(\lambda)+\rho^h_n(\lambda)+\sum_m(T_{n,m}*\rho_m)
(\lambda),
\end{equation}
%
where 
%
\begin{equation}
(T_{n,m}*\rho_m)(\lambda)\equiv\int_{-\infty}^{+\infty}T_{n,m}(\lambda-\lambda')
\rho_{m}(\lambda'). 
\end{equation}
%



%%%%%%%%%%%%%%%%%%%%%%%%%%%%%%%%%%%%%%%%%%%%%%%%%%%%%%%%%%%%%%%%%%%%%%%%%%%
\section{The conserved charges}

The $XXX$ chain exhibits an extensive number of local conserved charges $Q_n$. 
These are defined as 
%
\begin{equation}
\label{Q-def}
\left.Q_{j+1}\equiv\frac{i}{(j-1)!}\frac{d^j}{dy^j}\log\tau
(y)\right|_{y=i},
\end{equation}
%
where $y$ is a spectral parameter and $\tau(y)$ is the eigenvalue of the 
so-called transfer matrix in the Algebraic Bethe Ansatz framework. The 
analytic expression of $\tau(y)$ for a given eigenstate identified by 
a set of rapidities $\{\lambda_\alpha\}$ is given as 
%
\begin{equation}
\label{tau}
\tau(y)\equiv\Big(\frac{y+i}{2}\Big)^L\prod\limits_\alpha\frac{y-\lambda_\alpha-2i}
{y-\lambda_\alpha}+\Big(\frac{y-i}{2}\Big)^L\prod\limits_\alpha\frac{y-\lambda_\alpha
+2i}{y-\lambda_\alpha}.
\end{equation}
%
Using the string hypothesis and~\eref{Q-def} and~\eref{tau}, one obtains tha for 
a generic $n$-string with string center $\lambda$ the first few charges have the 
simple expressions
%
\begin{eqnarray}
Q_2=-\frac{2n}{\lambda^2+n^2}\\
Q_3=-\frac{4n\lambda}{(\lambda^2+n^2)^2}\\
Q_4=\frac{2n(n^2-3\lambda^2)}{(n^2+\lambda^2)^3}\\
Q_5=\frac{8n\lambda(n^2-\lambda^2)}{(n^2+\lambda^2)^4}\\
Q_6=-\frac{2n(5\lambda^4-10n^2\lambda^2+n^4)}{(n^2+\lambda^2)^5}
\end{eqnarray}
%

%%%%%%%%%%%%%%%%%%%%%%%%%%%%%%%%%%%%%%%%%%%%%%%%%%%%%%%%%%%%%%%%%%%%%%%%%%%
\subsection{N\'eel expectation values}
\label{Q-Neel}

The expectation values of the $XXX$ chain conserved charges over the N\'eel 
state are given as 

%%%%%%%%%%%%%%%%%%%%%%%%%%%%%%%%%%%%%%%%%%%%%%%%%%%%%%%%%%%%%%%%%%%%%%%%%%%
\section{Overlap with the Neel state}

Here we focus on the overlap with the zero-momentum N\'eel state $|N\rangle$, 
defined as 
%
\begin{equation}
\label{neel}
|N\rangle\equiv\frac{1}{\sqrt{2}}\Big(|\uparrow\downarrow\rangle^{\otimes L/2}
+|\downarrow\uparrow\rangle^{\otimes L/2}\Big).
\end{equation}
%
Here we restrict to the parity-invariant eigenstates. These are the only 
eigenstates with non-zero overlap with the Neel state. Parity-invariant 
eigenstates contain only pairs of rapidities with opposite sign.  

We denote the generic parity invariant eigenstate as $|\{\pm\lambda_j\}_{j=1}^m,
n_\infty\rangle$, where $m$ is the number of rapidity pairs, $N_{\infty}$ is the 
number of infinite rapidities, with $M=L/2=N_\infty+2m$, and $n_\infty\equiv 
N_\infty/L$ is the density of infinite rapidities. 

The overlap with the Neel state $|N\rangle$ reads 
%
\begin{equation}
\label{neel-ov}
\frac{\langle N|\{\pm\lambda_j\}_{j=1}^m,n_\infty\rangle}{|||\{\lambda_j\}_{j=1}^m,
n_\infty\rangle||}=\frac{\sqrt{2}N_{\infty}!}{\sqrt{(2N_\infty)!}}\left[\prod_{j=1}^m
\frac{\sqrt{\lambda_j^2+1}}{4\lambda_j}\right]\sqrt{\frac{\textrm{det}_m(G^+)}{
\textrm{det}_m(G^-)}}
\end{equation}
%
where 
%
\begin{equation}
\label{G-pm}
G^{\pm}_{jk}=\delta_{jk}\Big(NK_{1/2}(\lambda_j)-\sum\limits_{l=1}^mK_1^+(\lambda_j,
\lambda_l)\Big)+K_{1}^{\pm}(\lambda_j,\lambda_k),\quad\,j,k=1,\dots,m
\end{equation}
%
and 
%
\begin{equation}
\label{K-1}
K_1^\pm(\lambda,\mu)=K_1(\lambda-\mu)\pm K_1(\lambda+\mu)
\end{equation}
%
and 
%
\begin{equation}
\label{K-alpha}
K_\alpha(\lambda)\equiv\frac{8\alpha}{\lambda^2+4\alpha^2}
\end{equation}
%

%%%%%%%%%%%%%%%%%%%%%%%%%%%%%%%%%%%%%%%%%%%%%%%%%%%%%%%%%%%%%%%%%%%%%%%%%%%
\subsection{Reduced Neel overlap}

Here we consider the overlap formula for the Neel state~\eref{neel-ov} in the 
limit $L\to\infty$, assuming that the rapidities form perfect strings.  

In the case of perfect strings the matrices $G^{\pm}_{jk}$ become ill-defined. 
Precisely, $K_{1}^\pm(\lambda,\mu)$ diverges if $\lambda$ and $\mu$ are successive 
members of the same string, i.e., $|\lambda-\mu|=2i$.  

It is possible to rewrite~\eref{neel-ov} in terms of the string centers 
$\lambda_{n;\alpha}$ only. Here we restrict ourselves to rapidity configurations 
with no zero-momentum strings. Our results are not valid for zero-rapidity 
strings. These would require the knowledge of the precise form of the string 
deviations, i.e., the dependence of the string deviations on $L$, as it has 
been pointed out in Ref.~\cite{calabrese-2014}. 

It is convenient to split the indices $i,j$ of $G^\pm_{ij}$ as $i=(n,\alpha)$ 
$j=(m,\beta)$, with $n,m$ being the length of the strings and $\alpha,\beta$ 
labelling the string centers. 

The result reads 
%
\begin{equation}
\fl \frac{1}{2}G^+_{(n,\alpha)(m,\beta)}=\left\{\begin{array}{cc}
L\theta_n'(\lambda_{n;\alpha}) -\sum\limits_{(\ell,\gamma)\ne(n,\alpha)}\Big[\Theta'_{n,\ell}
(\lambda_{n;\alpha}-\lambda_{\ell;\gamma}) & \quad\textrm{if}\,(n,\alpha)=(m,\beta) \\
+\Theta'_{n,\ell}(\lambda_{n;\alpha}+\lambda_{\ell;\gamma})\Big] & \\ \\
\Theta'_{n,m}
(\lambda_{n;\alpha}-\lambda_{m;\beta})+\Theta'_{n,m}
(\lambda_{n;\alpha}+\lambda_{m;\beta}) & \quad\textrm{if}\,(n,\alpha)\ne(m,\beta)
\end{array}\right.
\end{equation}
%
Here $\theta_n'(x)\equiv d\theta_n(x)/dx=2n/(n^2+x^2)$ and $\Theta'(x)\equiv d\Theta(x)/dx$. 

For $G^-_{ij}$ one obtains 

\begin{equation}
\fl\frac{1}{2}G^-_{(n,\alpha)(m,\beta)}=\left\{\begin{array}{cc}
\fl(L-1)\theta'_n(\lambda_{n;\alpha})-2\sum\limits_{k=1}^{n-1}\theta'_k(\lambda_{n;\alpha})
& \textrm{if}\,(n,\alpha)= (m,\beta)\\
-\hspace{-.5cm}\sum\limits_{(\ell,\gamma)\ne(n,\alpha)}\Big[\Theta'_{n,\ell}
(\lambda_{n;\alpha}-\lambda_{\ell;\gamma})+\Theta'_{n,\ell}
(\lambda_{n;\alpha}+\lambda_{\ell;\gamma})\Big] \\\\
\Theta'_{n,m}
(\lambda_{n;\alpha}-\lambda_{m;\beta})-\Theta'_{n,m}
(\lambda_{n;\alpha}+\lambda_{m;\beta})) & \textrm{if}\,(n,\alpha)\ne(m,\beta)
\end{array}\right.
\end{equation}
%
Finally, the multiplicative prefactor in~\eref{neel-ov} for the generic $n$-string 
can be rewritten as 
%
\begin{equation}
\prod\limits_{a=1}^n\frac{\sqrt{(\lambda^a_{n;\alpha})^2+1}}{4\lambda^a_{n;\alpha}}=
\frac{1}{4^n}\left(\frac{\sqrt{n^2+\lambda^2_{n;\alpha}}}{\lambda_{n;\alpha}}
\prod\limits_{k=0}^{\lceil n/2\rceil-1}\frac{(2k)^2+\lambda^2_{n;\alpha}}{(2k+1)^2+
\lambda^2_{n;\alpha}}\right)^{{\mathcal P}},
\end{equation}
%
with ${\mathcal P}=+$ and ${\mathcal P}=-$ for even and odd strings, respectively. 



%%%%%%%%%%%%%%%%%%%%%%%%%%%%%%%%%%%%%%%%%%%%%%%%%%%%%%%%%%%%%%%%%%%%%%%%%%%
\section{Overlap with the Majumdar-Ghosh state}

The Majumdar-Ghosh state $|MG\rangle$ is defined as 
%
\begin{equation}
|MG\rangle\equiv \Big(\frac{|\uparrow\downarrow\rangle-|\downarrow\uparrow\rangle}
{\sqrt{2}}\Big)^{\otimes L/2}
\end{equation}
%
The overlap between the generic eigenstate of the $XXX$ chain and the Majumdar-Ghosh 
state is obtained fro the overlap with the Neel state as 
%
\begin{equation}
\label{mg-ov}
\langle MG|\{\lambda_j\}_{j=1}^M\rangle=\prod\limits_{j=1}^M\frac{1}{\sqrt{2}}
\Big(1-\frac{\lambda_j-i}{\lambda_j+i}\Big)\langle N|\{\lambda_j\}_{j=1}^M\rangle
\end{equation}
%
The mutliplicative factor in~\eref{ov-MG}, using the string hypothesis for the generic 
$n$-string is rewritten as 
%
\begin{equation}
\prod\limits_{j=1}^M\frac{1}{\sqrt{2}}
\Big(1-\frac{\lambda_j-i}{\lambda_j+i}\Big)=2^n\prod\limits_{k=0}^{\lfloor 
n/2\rfloor}\frac{1}{[\lambda_{n;\gamma}^2+(2k+(1-(-1)^n)/2)^2]^2}
\end{equation}
%

%%%%%%%%%%%%%%%%%%%%%%%%%%%%%%%%%%%%%%%%%%%%%%%%%%%%%%%%%%%%%%%%%%%%%%%%%%%
\section{Quench Action method}
\label{quench-action}

For integrable models the Quench-Action approach allows to extract exactly both 
the steady state after the quench and the relaxation dynamics to the steady state. 

In the thermodynamic limit the sum over the model eigenstates is recast into 
a functional integral over the so-called root distributions $\pmb{\rho}\equiv
\{\rho_n(\lambda)\}_{n=1}^\infty$ as
%
\begin{equation}
\label{eig-sum}
\sum\limits_{\alpha}\rightarrow\int{\mathcal D}\pmb{\rho}e^{S_{YY}(\pmb{\rho})}. 
\end{equation}
%
Here ${\mathcal D}\pmb{\rho}\equiv\prod_{n=1}^\infty{\mathcal D}\rho_n(\lambda)$, 
$\rho_n(\lambda)$ denote the distribution of the $n$-strings BGT roots, and 
$S_{YY}(\pmb{\rho})$ is the Yang-Yang entropy, which counts the number of 
Bethe states leading to the same $\pmb{\rho}$ in the thermodynamic limit. 
Formally, $\rho_n(\lambda)$  are defined as $\rho_n(\lambda)\equiv\lim_{L\to
\infty}(\lambda_{n;\gamma+1}-\lambda_{n;\gamma})^{-1}$. 

Using~\eref{eig-sum} the diagonal ensemble~\eref{d-ensemble} becomes 
%
\begin{equation}
\label{qa-d-ensemble}
\fl
\quad\langle{\mathcal O}\rangle=\int{\mathcal D}\pmb{\rho}(\lambda)
\exp\Big(2\Re\log\langle\Psi_0|\pmb{\rho}(\lambda)\rangle 
+S_{YY}(\pmb{\rho})\Big)\langle\pmb{\rho}(\lambda)|{\mathcal O}|\pmb{\rho}
(\lambda)\rangle
\end{equation}
%
Note that in~\eref{qa-d-ensemble} it is assumed that in the thermodynamic limit 
the eigenstate expectation values $\langle\alpha|{\mathcal O}|\alpha\rangle$ 
become smooth functionals  $\langle\pmb{\rho}(\lambda)|{\mathcal O}|\pmb{\rho}
(\lambda)\rangle$ of the root distributions $\pmb{\rho}$. 

Similarly, in~\eref{qa-d-ensemble} $\langle\Psi_0|\pmb{\rho}(\lambda)\rangle$ 
denotes the Bethe states overlap density with the prequench state $\Psi_0$. 

The strategy is to evaluate the functional integral in~\eref{qa-d-ensemble} 
using the saddle point approximation. 

One has then to minimize the functional ${\mathcal S}(\pmb{\rho})$ defined 
as 
%
\begin{equation}
L{\mathcal S}(\pmb{\rho})\equiv 2\Re\log\langle\pmb{\rho}|\Psi_0
\rangle+S_{YY}(\pmb{\rho}(\lambda)). 
\end{equation}
% 
under the constraint that the BGT equations hold. For the quench with initial 
state the N\'eel state the resulting saddle point root distributions 
$\rho_n^*(\lambda)$ can be obtained analytically. The first few root distributions 
are given as 
%
\begin{eqnarray}
\label{rho_sp}
\fl\quad\rho^*_1(\lambda)=\frac{8(4+\lambda^2)}{\pi(19+3\lambda^2)(1+6\lambda^2+
\lambda^4)}\\
\fl\quad\rho_2^*(\lambda)=\frac{8\lambda^2(9+\lambda^2)(4+3\lambda^2)}{\pi(2+\lambda^2)
(16+14\lambda^2+\lambda^4)(256+132\lambda^2+9\lambda^4)}\\
\fl\quad\rho_3^*=\frac{8(1+\lambda^2)^2(5+\lambda^2)(16+\lambda^2)(21+\lambda^2)}
{\pi(19+3\lambda^2)(9+624\lambda^2+262\lambda^4+32\lambda^6+\lambda^8)
(509+5\lambda^2(26+\lambda^2))}.
\end{eqnarray}
% 



%##################################################################
\begin{figure}[t]
\begin{center}
\includegraphics[width=.9\textwidth]{./draft_figs/Neel_overlaps}
\end{center}
\caption{ N\'eel state overlaps in the Heisenberg spin chain. (a) 
 The N\'eel state overlaps with the eigenstates of the Heisenberg 
 spin chain: Squared overlaps $|\langle\lambda|N\rangle|^2$ plotted 
 as a function of the eigenstates energy density $E/L$. Here $|\lambda
 \rangle$ denotes the generic eigenstate of the Heisenberg spin chain. 
 The data are for chains with length $26\le L\le 38$. The data are 
 obtained from a full scanning of the chains Hilbert space. 
 Eigenstates containing strings with zero-momentum components 
 are escluded. The overlaps decay exponentially with the chain size. 
 (b) Histograms of $-2\log|\langle\lambda|N\rangle|/L$. Different 
 histograms are for different chain sizes. The histogram bin width 
 is $\sim 2/L$. The $y$-axis is rescaled by a factor $10^5$ for 
 convenience. Note the peaking of the histograms around 
 $-2\log|\langle\lambda|N\rangle|\sim 0.5$ in the thermodynamic 
 limit {\bf really??}. 
}
\label{fig0:N-ov-overview}
\end{figure}
%##################################################################


%%%%%%%%%%%%%%%%%%%%%%%%%%%%%%%%%%%%%%%%%%%%%%%%%%%%%%%%%%%%%%%%%%%%%%%%%%%
\section{Overlap sum rules: the role of the zero-momentum strings}
\label{ov-sum-rules}

%##################################################################
\begin{figure}[t]
\begin{center}
\includegraphics[width=.9\textwidth]{./draft_figs/Neel}
\end{center}
\caption{ Overlap sum rules for the Neel state: The role of the zero-momentum 
 strings. (a) The overlap sum rule $\sum_{\lambda}|\langle\lambda|N\rangle|^2=1$, 
 with $|N\rangle$ the Neel state and $|\lambda\rangle$ the eigenstates  of 
 the $XXX$ spin chain. The $x$-axis shows $1/L$, with $L$ the chain length. 
 The circles are Bethe ansatz results for chains up to $L=36$. The results 
 are obtained via a full scanning of the chain Hilbert space. Only the 
 eigenstates with no zero-momentum strings are considered. The dash-dotted 
 line is a fit to $A/L^{1/2}+B/L$, with $A,B$ fitting parameters. Inset: 
 The same data as in the main Figure plotted versus $1/L^{1/2}$. (b) 
 The same as in (a) for the sum rule $\sum_{\lambda}|\langle
 \lambda|N\rangle|^2e(\lambda)=1/2$, with $e(\lambda)$  the 
 eigenstates energy density. 
}
\label{fig2-neel-sr}
\end{figure}
%##################################################################

In this section we investigate the role of the zero-momentum strings. We focus on 
conserved quantities sum rules. Specifically, given a generic conserved quantity 
(charge) $\hat Q$ we consider the trivial identity  
%
\begin{equation}
\label{sum}
Q_0\equiv\langle\Psi_0|\hat Q|\Psi_0\rangle=\sum\limits_{\lambda}|\langle\lambda|
\Psi_0\rangle|^2Q_\lambda, 
\end{equation}
%
where $Q_0$ is the charge expectation values over the initial state $|\Psi_0
\rangle$, $|\lambda\rangle$ denotes the generic eigenstate of the $XXX$ chain, 
and $Q_\lambda$ is the charge eigenvalue over the eigenstate. 

Here we restrict ourselves to the cases with $\hat Q=\mathbb{I}$ and $\hat 
Q={\mathcal H}$, considering both the case with $|\Psi\rangle=|N\rangle$ and 
$|\Psi_0\rangle=|MG\rangle$. 

For the N\'eel state, for any finite $L$ Eq.~\eref{sum} gives $Q_0=1,-1/2$ for 
$\hat{Q}=\mathbb{I}$ and $\hat Q={\mathcal H}$, respectively. 


Figure~\ref{fig2-neel-sr} plots the numerical results obtained via a full 
scanning of the Hilbert space of the Heisenberg chain. 

Precisely, the numerical data are obtained by generating all the possible string 
contents ${\mathcal S}$ compatible with the parity-invariance constraint

Panel (a) and (b) in the 
Figure plots the sum rule~\eref{sum} for $\hat Q=\mathbb{I}$ and $\hat Q={\mathcal 
H}$, respectively. The circles are obtained considering all the eigenstates of the 
Heisenberg spin chain with $L$ sites, excluding eigenstates containing 
zero-momentum strings. 

The total number of eigenstates $\widetilde Z_{Neel}$ containing no zero-momentum 
strings is given as 
%
\begin{equation}
\label{ztilde}
\widetilde Z_{Neel}=B\Big(\frac{L}{2},\frac{L}{4}\Big).
\end{equation}
%
The proof of\eref{ztilde} is reported in~\ref{app-2}. 

%##################################################################
\begin{figure}[t]
\begin{center}
\includegraphics[width=.9\textwidth]{./draft_figs/Dimer}
\end{center}
\caption{ Overlap sum rules for the Majumdar-Ghosh (MG) state. (a) The 
 overlap sum rule $\sum_{\lambda}|\langle\lambda|MG\rangle|^2=1$, 
 with $|MG\rangle$ the Majumdar-Ghosh state and $|\lambda\rangle$ 
 the eigenstates  of the $XXX$ spin chain. The $x$-axis shows $1/L$, 
 with $L$ the chain length. The circles are Bethe ansatz results for 
 chains up to $L=36$. The results are obtained via a full scanning of 
 the chain Hilbert space. Only the eigenstates with no zero-momentum 
 strings are considered. The dash-dotted line is a fit to $A/L+B/L^2$, 
 with $A,B$ fitting parameters. (b) The same as in (a) for the sum 
 rule $\sum_{\lambda}|\langle\lambda|MG\rangle|^2e(\lambda)=
 2/3$, with $e(\lambda)\equiv E/L$  the eigenstates energy density. 
}
\label{fig3-dimer-sr}
\end{figure}
%##################################################################


The largest chain size considered is $L=36$. The data are plotted versus $1/L$. 

Interestingly, both the sum rules (panels (a)(b)) are not saturated, as expected 
since the eigenstates corresponding to zero-momentum strings are excluded in the 
sum in the right-hand-side in~\eref{sum}. 

Interestingly, both sum rules exhibit vanishing behavior upon increasing the 
chain size. The dash-dotted lines in the Figures are fits to the behavior 
$A/L^{1/2}+B/L$, with $A,B$ fitting parameters, and perfectly describe the 
behavior of the data. 

The vanishing behavior as $\propto 1/L^{1/2}$ of the sum rules reflects the vanishing 
of the fraction of non-zero momentum string eigenstates in the thermodynamic limit ad 
%
\begin{equation}
\label{beh}
\frac{\widetilde Z_{Neel}}{Z_{Neel}}\propto\frac{4}{\sqrt{\pi L}}, 
\end{equation}
% 
with $Z_{Neel}$ being the total number of eigenstates corresponding to parity 
invariant BGT quantum numbers configurations. 

One should observe that the asymptotic, i.e., at large $L$, behavior~\eref{beh} 
is not generic, meaning that different initial states $|\Psi_0\rangle$ might 
give different behaviors. 

This is illustrated in Figure~\ref{fig3-dimer-sr}, focusing on the Majumdar-Ghosh 
(MG) state. Panels (a) and (b) plot the sum rules with $\hat Q=\mathbb{I}$ and 
$\hat Q={\mathcal H}$ (same as in Figure~\ref{fig2-neel-sr}). As for the N\'eel 
state only eigenstates containing no zero-momentum strings are consdidered. 
Note that their number $\widetilde Z_{MG}$ (see~\ref{app-2}) is now given as 
%
\begin{equation}
\widetilde Z_{MG}=B\Big(\frac{L}{2},\frac{L}{4}\Big)-B\Big(\frac{L}{2},
\frac{L}{4}-1\Big). 
\end{equation}
%
Similar to Figure~\ref{fig2-neel-sr}, due to the exclusion of the zero-momentum 
strings, the saturation rules exhibit vanishing behavior in the thermodynamic 
limit. However, in contrast with the N\'eel case (see Figure~\ref{fig2-neel-sr}), 
one has the behavior $Q_0\sim 1/L$, as confirmed by the fits (dash-dotted lines 
in Figure~\ref{fig3-dimer-sr}). 

Similar to the N\'eel case this reflects the same vanishing behavior of the 
eigenstates containing no zero-momentum strings $\widetilde Z_{MG}/Z_{MG}$ as 
%
\begin{equation}
\frac{\widetilde Z_{MG}}{Z_{MG}}=\frac{4}{4+L}. 
\end{equation}
%


%%%%%%%%%%%%%%%%%%%%%%%%%%%%%%%%%%%%%%%%%%%%%%%%%%%%%%%%%%%%%%%%%%%%%%%%%%%
\section{Validity of the string hypothesis for overlap calculations}
\label{string-ov}

Here we discuss the validity of the string hypothesis when calculating the 
overlaps between the eigenstates of the $XXX$ chain and the N\'eel state. 
Here we focus on the Heisenberg chain with $L=20$ sites. 
Figure~\ref{fig1-BGT-check} plots the squared overlaps $|\langle\lambda|
N\rangle|^2$ between the N\'eel state and the eigenstates of the chain. 
The overlaps are plotted against the eigenstate energy density $E/L
\in[-\log(2),0]$. The circles are exact diagonalization results for all 
the chain eigenstates ($382$ eigenstates), whereas the crosses denote the 
overlaps calculated using formula~\eref{neel-ov}, and the 
Bethe-Gaudin-Takahashi equations. Note that only the eigenstates with 
no zero-momentum strings are shown ($252$ eigenstates) in the Figure. 
Panel (a) in the Figure is an overview of all the results. Panels (b)-(d) 
correspond to zooming to the smaller overlap values $|\langle N|\lambda
\rangle|\lesssim 0.02$, $|\langle N|\lambda\rangle|\lesssim 0.002$, and 
$|\langle N|\lambda\rangle|\lesssim 10^{-5}$. 

Clearly, the overlaps decay rapidly upon increasing the energy density. 
This is expected since the $XXX$ Hamiltonian expectation value over 
the N\'eel state is $\langle N|{H}|N\rangle=-1/2$. Importantly, the 
agreement between the exact diagonalization results and the results 
obtained using the BGT equations~\eref{ba_eq} is excellent, confirming 
the validity of the string hypothesis. 


%##################################################################
\begin{figure}[t]
\begin{center}
\includegraphics[width=.9\textwidth]{./draft_figs/L20_BT_check}
\end{center}
\caption{ The squared overlap $|\langle N|\lambda\rangle|^2$ between the the 
 Neel state $|N\rangle$ and the eigenstates $|\lambda\rangle$ of the $XXX$ 
 chain with $L=20$ sites. Only non-zero overlaps are shown. In all the panels the 
 $x$-axis shows the eigenstate energy density $E/L$. The circles are the exact 
 diagonalization results for all the non-zero overlaps. The crosses are the Bethe 
 ansatz results obtained using the Bethe-Gaudin-Takahashi equations. The missing 
 crosses correspond to eigenstates containing zero-momentum strings. (a) Overview 
 of all the non-zero overlaps. (b)(c)(d) The same overlaps as in (a) zooming in 
 the regions $[0,0.2]$, $[0,0.020]$, and $[0,4\cdot 10^{-5}]$. The discrepancies 
 between the ED and the Bethe ansatz results are attributed to the string 
 deviations. 
}
\label{fig1-BGT-check}
\end{figure}
%##################################################################



%%%%%%%%%%%%%%%%%%%%%%%%%%%%%%%%%%%%%%%%%%%%%%%%%%%%%%%%%%%%%%%%%%%%%%%%%%%
\section{Monte Carlo implementation of the Quench-Action approach}
\label{mcqa-sec}


For the finite-size Heisenberg spin chain the Quench-Action expectation 
values can be obtained by sampling the eigenstates of the chain using 
Monte Carlo. One starts with an eigenstate of the $XXX$ chain with 
$N_{\infty}$ infinite rapidities and $M\equiv L/2-N_\infty$ finite ones. 
The state is identified by a parity-invariant Bethe quantum number 
configuration ${\mathcal C}$ and by the corresponding parity-invariant 
rapidities $\{\lambda\}$ as $|\lambda\rangle$. The string content 
associated with the finite rapidities is denoted as ${\mathcal S}$. The Monte 
Carlo procedure consists of four steps as follows: 
%
\begin{enumerate}
\item[\circled{1}] Choose a new number of finite rapidities $M'$ with 
probability
%
\begin{equation}
\label{PM}
{\mathcal P}(M')=\frac{\widetilde Z'_{Neel}(L,M')}{\widetilde{Z}_{Neel}(L)}. 
\end{equation}
%
\item[\circled{2}] Choose a new a new string content ${\mathcal S}'\equiv
\{s_1',\dots,s_{M'}'\}$ with probability ${\mathcal P}'(M',{\mathcal S}')$
%
\begin{equation}
\label{PS}
{\mathcal P}'(M',{\mathcal S}')=\frac{1}{\widetilde Z'_{Neel}(L,M')}
\prod_{n=1}^{M'}B\Big(\frac{L}{2}-\frac{1}{2}\sum\limits_{m=1}^{M'}t_{nm}
s'_m,s'_n\Big).
\end{equation}
%
\item[\circled{3}] Generate a new parity-invariant quantum number configuration 
${\mathcal C}'$ compatible with the ${\mathcal  S}'$ obtained in step $\circled{2}$. 
Solve the corresponding BGT equations~\eref{bt_eq}, finding a new eigenstate 
$|\lambda'\rangle$. 
\item[\circled{4}] Calculate the overlap $\langle\lambda'|N\rangle$ between the 
new eigenstate and the N\'eel state and accept the eigenstate with the Metropolis 
probability 
%
\begin{equation}
\label{metropolis}
{\mathcal P}''_{\lambda\to\lambda'}=\textrm{Min}\Big\{1,\exp\Big(-
2\Re({\mathcal E}'-{\mathcal E})\Big)\Big\}, 
\end{equation}
%
where ${\mathcal E}'\equiv-\log\langle\lambda'|N\rangle$, similarly for 
${\mathcal E}$, and $\Re$ denotes the real part. 
\end{enumerate}
%

Afte some necessary thermalization steps the sequence $1-4$ generates eigenstates 
of the $XXX$ chain distributed according to the Quench-Action probability 
distribution. 

For a generic observable ${\mathcal O}$, its Quench-Action expectation value 
$\langle{\mathcal O}\rangle_{QA}$ is obtained as the arithmetic average of the 
expectation value over the eigenstates sampled in the Monte Carlo as 
%
\begin{equation}
\langle{\mathcal O}\rangle_{QA}=\frac{1}{N_{mcs}}\sum\limits_{\lambda}
\langle\lambda|{\mathcal O}|\lambda\rangle, 
\end{equation}
%
with $N_{mcs}$ being the total number of Monte Carlo steps. 

%%%%%%%%%%%%%%%%%%%%%%%%%%%%%%%%%%%%%%%%%%%%%%%%%%%%%%%%%%%%%%%%%%%%%%%%%%%
\section{Monte Carlo Quench-Action approach: Numerical results}
\label{MCQA-nr}

%##################################################################
\begin{figure}[t]
\begin{center}
\includegraphics[width=.9\textwidth]{./draft_figs/QAMC_Obs_Neel}
\end{center}
\caption{The overlap sum rules for the Neel state: Numerical results 
 obtained using the Hilbert space Monte Carlo sampling approach. (a) 
 The energy sum rule $\langle N|Q_2|N\rangle/L=-1/2$, with $Q_2/L$ 
 the Hamiltonian density. We plot $\sum_{\lambda}|\langle\lambda|
 N\rangle|^2Q_2(\lambda)/L=1/2$, with $|\lambda\rangle$ the 
 eigenstates of the $XXX$ chain, versus the inverse chain length $1/L$. 
 The symbols are Monte Carlo data obtained by sampling the eigenstates 
 of the $XXX$ chain. The dash-dotted line is the expected result. The 
 dashed line is a fit to the behavior $-1/2+A/L+B/L^2$, with $A,B$ 
 fitting parameters. (b) The energy fluctuations sum rule $\sigma^2(Q_2)/
 L\equiv(\langle N|Q_2^2|N\rangle-\langle N|Q_2|N\rangle^2)/L=1/4$. The 
 horizontal line is the expected result. (c)(d) Same as in (a)(b) for 
 the charge $Q_4$ and its fluctuations. 
}
\label{fig5-neel-sr}
\end{figure}
%##################################################################


In this section we numerically demonstrate that the only effect of neglecting 
the zero-momentum strings are finite-size corrections. 

The validity of the Monte Carlo approach for simulating the Quench-Action 
results is demonstrated in Figure~\ref{fig5-neel-sr}. The Figure focuses on 
the N\'eel sum rules for the conserved charges $Q_2$ and $Q_4$. Specifically, 
panel (a) shows the sum rules for $\hat Q=\hat Q_2$, $\hat Q=\hat Q_4$, 
and the corresponding fluctuations $\sigma^2(Q_n)\equiv \langle N|\hat Q_n^2|
N\rangle-\langle N|\hat Q_n|N\rangle^2$. 
%
\begin{equation}
\langle N|\hat Q_n|N\rangle=\sum\limits_{\lambda}|\langle N|\lambda\rangle|^2
Q_n.
\end{equation}
%
Note that the N\'eel sum rule for the identity $\mathbb{I}$ is trivially satisfied, 
and it corresponds to the normalization of the Monte Carlo probability. 
Figure~\ref{fig5-neel-sr} (a) plots the N\'eel sum rule for the energy density $Q_2/L$. 
The circles are Monte Carlo data obtained using the procedure outlined in~\ref{mcqa-sec} 
for several Heisenberg spin chains. The largest chain simulated is for $L=56$ sites. 
The data correspond to a simulation with $\sim 10^7$ Monte Carlo step (mcs). 
The expected result $\langle N|\hat Q_2|N\rangle/L=-1/2$ is shown as dash-dotted 
line. 

In stark contrast with Figure~\ref{fig2-neel-sr} and Figure~\ref{fig3-dimer-sr}, 
while for small chains the sum rule is violated the data strongly suggest that 
in the thermodynamic limit the sum rule saturates the expected theoretical 
value. 

Furthermore, the data suggest the behavior $\propto 1/L$, as confirmed by the fit 
to $-1/2+A/L+b/L^2$ (dashed line in the Figure), with $A,B$ fitting parameters. 

The same qualitative behavior is demonstrated in panel (b) for the energy 
fluctuations. Interestingly, for $L=56$ the Monte Carlo result is compatible 
within the error bar with the expected asymptotic one. 

Finally, panels (c)(d) show the same results as for (a)(b) for the conserved 
charge $Q_4$. 

%%%%%%%%%%%%%%%%%%%%%%%%%%%%%%%%%%%%%%%%%%%%%%%%%%%%%%%%%%%%%%%%%%%%%%%%%%%
\section{The Quench-Action root distributions}
\label{MCQA-root}

%##################################################################
\begin{figure}[t]
\begin{center}
\includegraphics[width=.95\textwidth]{./draft_figs/Neel_rho}
\end{center}
\caption{ The Quench-Action steady state root ddistributions $\rho_1(x)$ 
 and $\rho_2(x)$: Monte Carlos results. (a) The histograms of the $1$-string 
 rapidities sampled in the Monte Carlo. The $x$-axis shows the $1$-string BGT 
 root $\lambda$. The data are for a chain with $L=56$ sites and $N_{mcs}\sim 
 10^7$ Monte Carlo steps. The data are divided by a factor $10^6$ for plotting 
 convenience. The width of the histogram bin is $\Delta\lambda\sim 0.07$. (b) 
 The same as in (a) for the $2$-string roots sampled in the Monte Carlo. (b) 
 The extracted $1$-string root distribution $\rho_1(\lambda)$ plotted versus 
 $\lambda$ for two chains with $L=48$ and $L=56$ (diamond and circles, respectively). 
 The full line is the Quench-Action analytic result in the thermodynamic limit. (d) 
 The same as in (c) for the $2$-string root density $\rho_2(\lambda)$. In both 
 (c)(d) the oscillations are finite-size effects, whereas the error bars 
 show the statistical Monte Carlo error. 
}
\label{fig6-neel-roots}
\end{figure}
%##################################################################

The BGT root distributions $\rho_n(\lambda)$ identifying the Quench-Action saddle 
point can be extracted from the Monte Carlo simulation as shown in~\cite{alba-2015}. 

Figure~\ref{fig6-neel-roots} displays the first two root distributions 
$\rho_1(\lambda)$ (panel (a)) and $\rho_2(\lambda)$ (panel (b)), plotted agains 
the BGT root $\lambda$. The data are the histograms of the BGT roots sampled 
in the Monte Carlo. The data are for a chain with $L=56$ sites and 
$N_{mcs}\sim 10^7$ Monte Carlo steps. All the data are divided by a factor 
$10^6$ for plotting convenience. The width of the histogram bins $\Delta\lambda$ 
is $\Delta\lambda\approx0.02$ and $\Delta\lambda\approx0.001$ for $\rho_1(\lambda)$ 
and $\rho_2(\lambda)$, respectively. 

The histogram fluctuations are due to $N_{mcs}$ being finite, and are expected 
to vanish in the limit $N_{mcs}\to\infty$. 

The continuous lines is the expected analytic result in the thermodynamic limit. 

Clearly, $\rho_1(\lambda)$ is in good agreement with the Monte Carlo data in the 
whole range $-2\le\lambda\le2$ considered. 
On the other hand, larger deviations from the theoretical result are present 
for $\rho_2(\lambda)$. These deviations are larger on the tails of the 
distribution. This is expected since large $\lambda$ correspond to large 
quasimomenta, which are more sensitive to the finite size of the chain. 



%%%%%%%%%%%%%%%%%%%%%%%%%%%%%%%%%%%%%%%%%%%%%%%%%%%%%%%%%%%%%%%%%%%%%%%%%%%
\subsection{Counting of eigenstates with non-zero Neel overlap}

We numerically checked that the number of states with non-zero overlap 
with the Neel state is given as 
%
\begin{equation}
\label{Neel-count}
Z_{Neel}=2^{\frac{L}{2}-1}+\frac{1}{2}B\left(\frac{L}{2},
\frac{L}{4}\right)+1,
\end{equation}
%
with $B(x,y)$ denoting the binomial coefficient. The contribution $1$ 
accounts for the ferromagnetic state. Here 
we assumed that $L$ is divisible by four. Here $Z_N$ is be obtained 
as the total number of parity-invariant Bethe-Gaudin-Takahashi quantum 
numbers. 

After excluding the zero-momentum strings the total number of states 
with non-zero overlap with the Neel state is 
%
\begin{equation}
\label{neel-ov-count}
\widetilde {Z}_{Neel}=\Big(\frac{L}{2},\frac{L}{4}\Big)
\end{equation}
%
Importantly, the fraction of eigenstates corresponding to non-zero momentum 
strings is vanishing in the thermodynamic limit as 
%
\begin{equation}
\frac{\widetilde Z_{Neel}}{Z_{Neel}}\propto\frac{4}{\sqrt{\pi L}}.
\end{equation}
%

\appendix

%%%%%%%%%%%%%%%%%%%%%%%%%%%%%%%%%%%%%%%%%%%%%%%%%%%%%%%%%%%%%%%%%%%%%%%%%%%
\section{Eigenstates with nonzero N\'eel overlap: eigenstates counting and 
string content}
\label{app-1}

Here we prove that the total number of eigenstates with N\'eel nozero overlap 
$Z_{Neel}(L)$for a chain of length $L$ is given as 
%
\begin{equation}
\label{N-count}
Z_{Neel}=2^{\frac{L}{2}-1}+\frac{1}{2}B\Big(\frac{L}{2},\frac{L}{4}\Big)+1. 
\end{equation}
%
For simplicity here we restrict ourselves to the situation with $L$ divisible by 
four. The strategy to prove~\eref{N-count} is to count all the possible parity-invariant 
BGT quantum numbers configurations. Let us consider the sector with fixed number of 
particles $M$, and a generic string content ${\mathcal S}=\{s_1,s_2,\dots,s_{M}\}$. 
Here $s_n$ is the number of $n$-strings, and one has the constraint $\sum_k ks_k=M$. 

It is straightforward to check that total number of parity-invariant quantum number 
pairs ${\mathcal N}_n(L,{\mathcal S})$ in the $n$-string sector is given as 
%
\begin{equation}
\label{NnLS}
{\mathcal N}_n(L,{\mathcal S})=\Big\lfloor\frac{L}{2}-\frac{1}{2}
\sum_{m=1}^{M}t_{nm}s_m\Big\rfloor.
\end{equation}
%
where $t_{nm}\equiv 2\textrm{Min}(n,m)-\delta_{n,m}$. The number of parity-invariant 
quantum number configurations (i.e., eigenstates) ${\mathcal N}(L,{\mathcal S})$ 
compatible with string content ${\mathcal S}$ is obtained by choosing in all the 
possible ways the parity-invariant quantum number pairs independently in each 
$n$-string sector, which implies that    
%
\begin{equation}
\label{NLS}
{\mathcal N}(L,{\mathcal S})=\prod_{m=1}^{M} B\left({\mathcal N}_m,\left\lfloor
\frac{s_m}{2}\right\rfloor\right).
\end{equation}
%
Here the product is because each string sector is treated independently, while the 
factor $1/2$ in $s_m/2$ is because since all quantum numbers are organized in pairs, 
only half of the quantum numbers have to be specified. Note that in each $n$-string 
sector only one zero momentum (i.e., zero quantum number) string is allowed, due to 
the Pauli principle. Moreover, $s_m$ is odd (even) only if the zero momentum string 
is (not) present. The floor function $\lfloor\cdot\rfloor$ in~\eref{NLS} reflects 
that the quantum number of zero-momentum strings is fixed. 

We now consider the string configurations with particle number $0\le\ell\le M$ and 
fixed number of strings $1\le q\le M/2$. Note that the maximul allowed string length 
is $M/2$ beacause of parity invariance. Note also that in determining $q$ strings of 
different length are treated equally. Clearly, one has that $\sum_m s_m=q$. 
For a given fixed pair $\ell,q$ the total number of quantum number configurations 
is given as 
%
\begin{equation}
\label{NLlq}
{\mathcal N}'(L,\ell,q)=\sum\limits_{\{\{s_m\}\,:\, \sum m s_m=\ell, \sum s_m=q\}}
{\mathcal N}(L,{\mathcal S}),
\end{equation}
%
where the sum is over the content $\{s_m\}_{m=1}^M$ compatible with the constraints 
$\sum_m s_m=q$ and $\sum_m m s_m=\ell$. The strategy is to write a recursive relation 
for ${\mathcal N}'(L,\ell,q)$. To this purpose it is useful to consider the shifted 
string content ${\mathcal S}'$ defined as  
%
\begin{equation}
{\mathcal S}'\equiv \{s_{m+1}\}\quad\textrm{with}\, s_m\in{\mathcal S},\,\forall m.
\end{equation}
%
Using the definition of $t_{ij}$, it is straightforward to derive that  
%
\begin{equation}
t_{ij}=t_{i-1,j-1}+2,
\end{equation}
%
which implies that ${\mathcal N}_n(L,{\mathcal S})$ (see~\eref{NnLS}) satisfies the 
recursive equation 
%
\begin{equation}
{\mathcal N}_n(L,{\mathcal S})={\mathcal N}_{n-1}(L-2q,{\mathcal S}'). 
\end{equation}
%
After substituting in~\eref{NLS} one obtains 
%
\begin{equation}
\label{NLSr}
{\mathcal N}(L,{\mathcal S})=B\Big({\mathcal N}_1(L,{\mathcal S}),\Big\lfloor \frac{s_1}{2}
\Big\rfloor\Big){\mathcal N}(L-2q,{\mathcal S}'). 
\end{equation}
%
Finally, after substituting~\eref{NLSr} in~\eref{NLlq}, one obtains a recursive relation 
for ${\mathcal N}'(L,\ell,q)$ as 
%
\begin{equation}
\label{NpLlq}
{\mathcal N}'(L,\ell,q)=\sum_{s=0}^{q-1}B\Big(\frac{L}{2}-q+\Big\lfloor\frac{s}{2}\Big\rfloor,
\left\lfloor\frac{s}{2}\right\rfloor\Big){\mathcal N}'\left(L-2q,\ell-q,
q-s\right), 
\end{equation}
%
with the constraint that when $\ell=q$ one has 
%
\begin{equation}
{\mathcal N}'(L,q,q)=B\Big(\Big\lfloor\frac{L-q}{2}\Big\rfloor,\Big\lfloor\frac{q}{2} 
\Big\rfloor\Big).
\end{equation}
%
This is obtained by observing that if $\ell=q$ only $1$-strings are allowed and~\eref{NnLS} 
gives ${\mathcal N}_n(L,{\mathcal S})=\lfloor (L-q)/2\rfloor$. 

It is straightforward to check that for even $q$ the ansatz 
%
\begin{equation}
{\mathcal N}'(L,\ell,q)=\frac{q}{\ell}B\Big(\frac{L-\ell}{2},\frac{q}{2}\Big)
B\Big(\frac{\ell}{2},\frac{q}{2}\Big),
\end{equation}
% 
satisfies~\eref{NpLlq}. For odd $q$ the solution of~\eref{NpLlq} is 
%
\begin{equation}
{\mathcal N}'(L,\ell,q)=\frac{\ell-q+1}{\ell}B\Big(\frac{L-\ell}{2},\frac{q-1}{2}
\Big)B\Big(\frac{\ell}{2},\frac{q-1}{2}\Big).
\end{equation}
%
The number of eigenstates in the sector with $\ell$ particles with nonzero 
N\'eel overlap $Z'_{Neel}(L,\ell)$ are obtained by summing over all possible values 
of $q$ as 
%
\begin{equation}
\label{sum1}
Z'_{Neel}(L,\ell)=\sum\limits_{q=1}^\ell {\mathcal N}'(L,\ell,q).
\end{equation}
%
It is convenient to split the summation in~\eref{sum1} considering odd values of $q$ 
and even $q$ separately. For odd $q$ one obtains 
%
\begin{equation}
\sum\limits_{k=0}^{\ell/2-1} {\mathcal N}'(L,\ell,2k+1)=B\Big(\frac{L}{2}-1,
\frac{\ell}{2}-1\Big),
\end{equation}
%
while for even $q$ one has 
%
\begin{equation}
\sum\limits_{k=0}^{\ell/2} {\mathcal N}'(L,\ell,2k)=B\Big(\frac{L}{2}-1,
\frac{\ell}{2}\Big). 
\end{equation}
%
Putting everything together one obtains 
%
\begin{equation}
\label{N-count-p}
Z'_{Neel}(L,\ell)=B\Big(\frac{L}{2}-1,
\frac{\ell}{2}-1\Big)+B\Big(\frac{L}{2}-1,
\frac{\ell}{2}\Big). 
\end{equation}
%
The total number of eigenstates with nonzero N\'eel overlap $Z_{Neel}(L)$ 
(cf.~\eref{N-count}) is obtained from~\eref{N-count-p} by summing over 
the allowed values of $\ell=2k$ with $k=0,1,\dots,\ell/2$. 

Note that the total number $Z_{MG}$ of parity-invariant eigenstates having non 
zero overlap with the Majumdar-Ghosh state is obtained from Eq~\eref{N-count-p} 
replacing $\ell=L/2$, to obtain 
%
\begin{equation}
\label{p-inv-mg}
Z_{MG}=B\Big(\frac{L}{2}-1,\frac{L}{4}-1\Big)+B\Big(\frac{L}{2}-1,\frac{L}{4}
\Big). 
\end{equation}
%
Physically, this is due to the fact that the 
Majumdar-Ghosh state is invariant under $SU(2)$ rotations, which implies 
that only eigenstates with zero total spin $S=0$ can have non zero overlap. 


%%%%%%%%%%%%%%%%%%%%%%%%%%%%%%%%%%%%%%%%%%%%%%%%%%%%%%%%%%%%%%%%%%%%%%%%%%%
\section{Excluding the zero-momentum strings}
\label{app-2}

Here we demonstrate that the total number of eigenstates with nonzero N\'eel overlap, 
which do not contain zero-momentum strings, $\widetilde Z_{Neel}(L)$ is given as 
%
\begin{equation}
\widetilde Z_{Neel}(L)=B\Big(\frac{L}{2},\frac{L}{4}\Big). 
\end{equation}
%
Given a generic $M$-particle eigenstate of the $XXX$ chain, due to parity invariance, 
if one excludes the zero-momentum strings only $n$-strings with length $n\le M/2$ 
are allowed. Similarly, the string content is of the form $\widetilde{\mathcal S}
\equiv\{\tilde s_1,\dots,\tilde s_{M/2}\}$, i.e., $\tilde s_m=0$ $\forall m>M/2$.
Note that due to parity invariance and to the exclusion of the zero-momentum strings 
one has that $\tilde s_m$ is always an even integer. Clearly one has $\sum_{m=1}^{M/2}
m \tilde s_m=M$. 

The total number of parity-invariant quantum numbers $\widetilde{\mathcal N}_n$ in the 
$n$-string sector is given as  
%
\begin{equation}
\widetilde{\mathcal N}_n(L,\widetilde{\mathcal S})=\frac{L}{2}-\frac{1}{2}
\sum_{m=1}^{M/2}t_{nm}\tilde s_m.
\end{equation}
%
The proof now proceeds as in~\ref{app-1}. One can define the total number of eigenstates 
with nonzero N\'eel overla in the sector with $\ell$ particles and $q$ different strings as 
$\widetilde{\mathcal N}'(L,\ell,q)$. Note that due to parity invariance and the exclusion of 
zero-momentum strings, $q$ must be even. It is straigtforward to show that $\widetilde
{\mathcal N}'(L,\ell,q)$ obeys the recursive relation
%
\begin{equation}
\label{NpLlq-1}
\widetilde{\mathcal N}'(L,\ell,q)=\sum_{s=0}^{q/2-1}B\Big(\frac{L}{2}-q+s,s\Big)\widetilde
{\mathcal N}'\Big(L-2q,\frac{\ell-q}{2},\frac{q}{2}-s\Big),
\end{equation}
% 
with the constraint
%
\begin{equation}
\widetilde{\mathcal N}'(L,1,1)=\frac{L}{2}-1. 
\end{equation}
%
It is straightforward to check that the solution of~\eref{NpLlq-1} is given as 
%
\begin{equation}
\widetilde{\mathcal N}'(L,\ell,q)=\frac{L-2\ell+2}{L-\ell+2}B\Big(\frac{L-\ell}{2}+1,q\Big)
B\Big(\frac{\ell}{2}-1,\frac{q}{2}-1\Big).
\end{equation}
%
After summing over the allowed values of $q=2k$ with $k=1,2,\dots,\ell/2$ one obtains 
the total number of eigenstaets with nonzero N\'eel overlap at fixed number of 
particles $\ell$ $\widetilde Z_{Neel}'(L,\ell)$ as 
%
\begin{equation}
\label{neel-fi}
\widetilde Z_{Neel}'(L,\ell)=B\Big(\frac{L}{2},\frac{\ell}{2}\Big)-
B\Big(\frac{L}{2},\frac{\ell}{2}-1\Big).
\end{equation}
%
Summing over $\ell$ one obtains~\eref{neel-ov-count}.
Similar to~\eref{p-inv-mg} the total number of eigenstates $\widetilde Z_{MG}$ 
with no zero-momentum strings having non-zero overlap with the Majumdar-Ghosh 
state is obtained from~\eref{neel-fi} replacing $\ell\to L/2$, to obtain 
%
\begin{equation}
\label{mg-fi}
\widetilde Z_{MG}=B\Big(\frac{L}{2},\frac{L}{4}\Big)-B\Big(\frac{L}{2},\frac{L}{4}-1
\Big). 
\end{equation}
%
Interestingly, using~\eref{p-inv-mg} and~\eref{mg-fi}, one obtains that the ratio 
$\widetilde Z_{MG}/Z_{MG}$ is given as 
%
\begin{equation}
\frac{\widetilde Z_{MG}}{Z_{MG}}=\frac{4}{4+L}. 
\end{equation}
%


%%%%%%%%%%%%%%%%%%%%%%%%%%%%%%%%%%%%%%%%%%%%%%%%%%%%%%%%%%%%%%%%%%%%%%%%%%%
\section{Exact N\'eel and Majumdar-Ghosh overlaps for a small Heisenberg chain} 
\label{app-L12}

In this section we provide exact diagonalization results for the overlap of both the 
N\'eel state and the Majumdar-Ghosh (MG) state with all the eigenstates of the Heisenberg 
spin chain with $L=12$ sites. We also provide the corresponding results obtained 
using the string hypothesis and the overlap formulas~\eref{neel-ov} and~\eref{mg-ov}, 
restricting ourselves to eigenstates with no zero-momentum strings. 

%%%%%%%%%%%%%%%%%%%%%%%%%%%%%%%%%%%%%%%%%%%%%%%%%%%%%%%%%%%%%%%%%%%%%%%%%%%
\subsection{N\'eel overlap}
\label{app-neel}

The overlaps between all the eigenstates of the Heisenberg spin chain and the N\'eel 
state are reported in Table~\ref{table:neel}. The first column in the Table shows 
the string content ${\mathcal S}\equiv\{s_1,\dots,s_M\}$, with $M$ being the number 
of finite rapidities. The number of infinite rapidities $N_{\infty}=L/2-M$ is also 
reported. Note that eigenstates containing infinite rapidities correspond to 
different $S_z$ eigenvalue. The second column shows 
$2I_n^+$, with $I_n$ the Bethe-Gaudin-Takahashi quantum number identifying the 
BGT rapidity of the $n$-string. Due to the parity invariance only the positive 
quantum numbers are reported. The total number of independent strings, i.e., 
$q\equiv\sum_js_j$, is reported in the third column. The fourth column is the 
eigenstate's energy eigenvalue $E$. The last two columns show the squared N\'eel 
overlaps and the corresponding result obtained using the Bethe-Gaudin-Takahashi 
equations, respectively. In the last column only the case with no zero-momentum 
strings is considered. The deviations from the exact diagonalization results (digits 
with different colors) have to be attributed to the string hypothesis. Notice that  
the overlap between the N\'eel state and the $S_z=0$ eigenstate in the sector with 
maximal total spin $S=L/2$ (first column in Table~\ref{table:neel}), is given 
analytically as $2/B(L,L/2)$, with $B(x,y)$ the Newton binomial. 

%%%%%%%%%%%%%%%%%%%%%%%%%%%%%%%%%%%%%%%%%%
\begin{table}[h]
\scriptsize
\centering
Bethe states with nonzero N\'eel overlap ($L=12$)\\[1ex]
\begin{tabular}{rrrrrr}
\toprule
String content & $2I^+_n$ & $q$ & E & $|\langle\lambda|N\rangle|^2$ (exact) & $|\langle
\lambda|N\rangle|^2$ (BGT) \\[0.3em]
\toprule
6 inf & - & - & $0$ & $0.002164502165$ & $0.002164502165$\\
\midrule
\{2,0\}\, 4 inf &$1_1$ & $2$ & $-3.918985947229$ & $0.096183409244$ & $0.096183409244$\\
 &$3_1 $ & & $-3.309721467891$ & $0.011288497947$ &             $0.011288497947$\\
 &$5_1 $ & & $-2.284629676547$ & $0.004542580506$ &             $0.004542580506$\\
 &$7_1 $ & & $-1.169169973996$ & $0.002752622983$ &             $0.002752622983$\\
 &$9_1 $ & & $-0.317492934338$ & $0.002116006203$ &             $0.002116006203$\\
\midrule
\{4,0,0,0\}\, 2 inf &$1_1 3_1 $ & $4$ & $-7.070529325964$ & $0.310133033838$ &$0.310133033838$\\
  &$1_1 5_1 $ & & $-5.847128730477$ & $0.129277023687$ &           $0.129277023687$\\
  &$ 1_1 7_1$ & & $-4.570746557876$ & $0.085992436024$ &           $0.085992436024$\\
  &$ 3_1 5_1$ & & $-5.153853093221$ & $0.015256395523$ &           $0.015256395523$\\
  &$3_1 7_1 $ & & $-3.916336243695$ & $0.010091113504$ &           $0.010091113504$\\
  &$5_1 7_1 $ & & $-2.817696043731$ & $0.004059780228$ &           $0.004059780228$\\
\midrule
\{0,2,0,0\}\, 2 inf &$1_2 $ & $2$ & $-1.905667167442$ & $0.001207238321$ & $0.0012072{\color{red}45406}$\\
  &$3_2 $ & & $-1.368837200825$ & $0.002340453815$ &            $0.0023{\color{red}25724713}$\\
  &$5_2 $ & & $-0.681173793635$ & $0.001921010489$ &            $0.0019{\color{red}39001396}$\\
\midrule
\{1,0,1,0\}\, 2 inf &$0_1 0_3$ & $2$ & $-2.668031843135$ & $0.034959609810$ & -\\
\midrule
\{6,0,0,0,0,0\}\, 0 inf &$1_1 3_1 5_1$ & $6$ & $-8.387390917445$ & $0.153412152966$ & $0.153412152966$\\
\midrule
\{2,2,0,0,0,0\}\, 0 inf &$1_1 1_2$ & $4$ & $-5.401838225870$ & $0.040162686361$ & $0.04{\color{red}1042488913}$\\  
&$3_1 1_2 $ & & $-4.613929948329$ & $0.004636541934$ & $0.004{\color{red}730512604}$\\
  &$5_1 1_2 $ &  & $-3.147465758841$ & $0.001335522556$ & $0.00133{\color{red}7334035}$\\
\midrule
\{3,0,1,0,0,0\}\, 0 inf &$0_1 2_1 0_3$ & $4$ & $-6.340207488736$ & $0.052743525774$ & -\\
  &$0_1 4_1 0_3$ & & $-5.203653009936$ & $0.015022005621$ & - \\
  &$0_1 6_1 0_3$ & & $-3.788693957250$ & $0.011144489334$ & - \\
\midrule
\{1,0,0,0,1,0\}\, 0 inf &$0_1 0_5$ & $2$ & $-2.444293750583$ & $0.005887902992$ & - \\
\midrule
\{0,0,2,0,0,0\}\, 0 inf &$1_3$ & $2$ & $-1.111855930538$ & $0.001342476001$ & $0.0013{\color{red}84980817}$ \\
\midrule
\{0,1,0,1,0,0\}\, 0 inf &$0_2 0_4$ & $2$ &  $-1.560671012472$ & $0.000026982174$ & - \\
\bottomrule
\end{tabular}
\caption{All Bethe states for $L=12$ having nonzero overlap with the zero-momentum N\'eel state. 
 The first column shows the string content of the Bethe states, including the number of infinite 
 rapidities. The second and third column show $2I_n^+$, with $I_n^+$ the BGT quantum numbers 
 identifying the different states, and the number $q$ of independent strings. In the second 
 column only the positive BGT numbers are shown. The fourth column is the Bethe state eigenenergy. 
 Finally, the last two columns show the exact overlap with the N\'eel state and the approximate 
 result obatained using the BGT equations. In the last column Bethe states containing zero-momentum 
 strings are excluded. Deviations from the exact result (digits with different colors) are 
 attributed to the string hypothesis. 
}
\label{table:neel}
\end{table}
%%%%%%%%%%%%%%%%%%%%%%%%%%%%%%%%%%%%%%%%%%



%%%%%%%%%%%%%%%%%%%%%%%%%%%%%%%%%%%%%%%%%%%%%%%%%%%%%%%%%%%%%%%%%%%%%%%%%%%
\subsection{Majumdar-Ghosh overlap}
\label{app-mg}
The overlap between the Heisenberg chain eigenstates with the Majumdar-Ghosh state are shown in 
Table~\ref{table:mg}. The conventions on the representation of the eigenstates is the same as in 
Table~\ref{table:neel}. Note that in contrast with the N\'eel state, only the eigenstates with 
zero total spin $S=0$ have non zero overlap, i.e., no eigenstates with infinite rapidities are 
present, which reflect that the Majumdar-Ghosh state is unvariant under $SU(2)$ rotations. 

%%%%%%%%%%%%%%%%%%%%%%%%%%%%%%%%%%%%%%%%%%
\begin{table}[h]
\scriptsize
\centering
Bethe states with nonzero N\'eel overlap ($L=12$)\\[1ex]
\begin{tabular}{rrrrrr}
\toprule
String content & $2I^+_n$ & $q$ & E & $|\langle\lambda|MG\rangle|^2$ (exact) & $|\langle\lambda|MG\rangle|^2$ (BGT) \\[0.3em]
\toprule
\{6,0,0,0,0,0\} &$1_1 3_1 5_1$ & $6$ & $-8.387390917445$ & $0.716615769224$ & $0.716615769224$\\
\midrule
\{2,2,0,0,0,0\} &$1_1 1_2$ & $4$ & $-5.401838225870$ & $0.055624700196$ & $0.05{\color{red}4033366543}$\\  
&$3_1 1_2 $ & & $-4.613929948329$ & $0.005687428810$ & $0.005{\color{red}582983043}$\\
&$5_1 1_2 $ &  & $-3.147465758841$ & $0.002107475934$ & $0.002107{\color{red}086933}$\\
\midrule
\{3,0,1,0,0,0\} &$0_1 2_1 0_3$ & $4$ & $-6.340207488736$ & $0.205891158647$ & -\\
  &$0_1 4_1 0_3$ & & $-5.203653009936$ & $0.038832154450$ & - \\
  &$0_1 6_1 0_3$ & & $-3.788693957250$ & $0.006019410923$ & - \\
\midrule
\{1,0,0,0,1,0\} &$0_1 0_5$ & $2$ & $-2.444293750583$ & $0.000129601311$ & - \\
\midrule
\{0,0,2,0,0,0\} &$1_3$ & $2$ & $-1.111855930538$ & $0.000011727787$ & $0.00001{\color{red}2785580}$\\
\midrule
\{0,1,0,1,0,0\} &$0_2 0_4$ & $2$ &  $-1.560671012472$ & $0.000330572718$ & - \\
\bottomrule
\end{tabular}
\caption{All Bethe states for $L=12$ having nonzero overlap with the zero-momentum Majumdar-Ghosh (MG) 
 state. The first column shows the string content of the Bethe states. The second and third column show 
 $2I_n^+$, with $I_n^+$ the BGT quantum numbers identifying the different states, and the number $q$ 
 of independent strings. In the second column only the positive BGT numbers are shown. Note that, in 
 contrast to Table~\ref{table:neel} no states with infinite rapidities are present. The fourth column 
 is the Bethe state eigenenergy. Finally, the last two columns show the exact overlap with the MG state 
 and the approximate result obatained using the BGT equations. In the last column Bethe states containing 
 zero-momentum strings are excluded. Deviations from the exact result (digits with different colors) 
 are attributed to the string hypothesis. 
}
\label{table:mg}
\end{table}
%%%%%%%%%%%%%%%%%%%%%%%%%%%%%%%%%%%%%%%%%%




%%%%%%%%%%%%%%%%%%%%%% REFERENCES %%%%%%%%%%%%%%%%%%%%%%%%%%%%%%%%%%%%%%%%%%%%%%%%
\begin{thebibliography}{99}

\bibitem{calabrese-2014}
P.~Calabrese and P.~Le Doussal, J.\ Stat.\ Mech.\ (2014) P05004. 

\bibitem{alba-2015}
V.~Alba, arXiv:1507.06994.

\end{thebibliography}


\end{document}


