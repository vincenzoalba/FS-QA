\documentclass[12pt,a4paper,final]{iopart} 
\usepackage{graphicx}
%\pdfoutput=1
\usepackage{amsfonts, dsfont}

\usepackage{leftidx}
\usepackage{cite}
\usepackage{booktabs}


\usepackage{verbatim}
\usepackage{color}
\usepackage[colorlinks,bookmarks=false,citecolor=blue,
linkcolor=red,urlcolor=blue]{hyperref}
\usepackage{times}




%%%%%%%%%%%%%   useful shortcuts %%%%%%%%%%%%%%%%%%%%%%%%%%%%%%%%%



%%%%%%%%%%%%%%%%%%%%%%%%%%%%%%%%%%%%%%%%%%%%%%%%%%%%%%%





\begin{document}



\begin{table}[h]
\scriptsize
\centering
Bethe states with nonzero M-G overlap ($L=12$)\\[1ex]
\begin{tabular}{rrrr}
String content & Energy (ED) & $|\langle \{\lambda\}| \Psi_0 \rangle|^2$ (ED) & $|\langle \{\lambda\}| \Psi_0 \rangle|^2$ (B-T) \\[0.3em]
\toprule
$1_1 3_1 5_1$ & $-8.38739$  & $0.716616$     &  $0.7166157692239$\\
              & $-6.34021$  & $0.205891$     &  \\
$5_1 1_2$     & $-5.40184$  & $0.0556247$    & $0.05403336654338$\\
              & $-5.20365$  & $0.0388322$    & \\
$3_1 1_2$     & $-4.61393$  & $0.00568743$   & $0.005582983043235$\\
              & $-3.78869$  & $0.00601941$   & \\
$1_1 1_2$     & $-3.14747$  & $0.00210748$   & $0.0021070869333835$\\
              & $-2.44429$  & $0.000129601$  & \\
              & $-1.56067$  & $0.000330573$  & \\
$1_3$         & $-1.11186$  & $0.0000117278$ & $0.000012785579923275$\\
\bottomrule
\end{tabular}
\caption{All Bethe states for $L=12$ with nonzero overlap with the zero-momentum 
 M-G state. The first column show the string content of the eigenstate. The second 
 and third columns show the exact diagonalization resutls for the energy and the 
 overlap, respectively. The last colum is the overlap obtained in the Bethe ansatz 
 approach using the Bethe-Takahashi equations.
}
\label{table:RV:sumruleN12}
\end{table}


\begin{table}[h]
\scriptsize
\centering
Bethe states with nonzero M-G overlap ($L=16$)\\[1ex]
\begin{tabular}{rrrr}
String content & Energy (ED) & $|\langle \{\lambda\}| \Psi_0 \rangle|^2$ (ED) & $|\langle \{\lambda\}| \Psi_0 \rangle|^2$ (B-T) \\[0.3em]
\toprule
$7_1 5_1 3_1 1_1$ & $-11.1423$ & $0.517742$         & $0.5177418283152$\\
& $-9.59129$ & $0.244845$         &\\
$7_1 5_1 1_2$ & $-8.81424$ & $0.0727096$        & $0.07100180464371$\\
& $-8.56579$ & $0.082852$         &\\
$7_1 3_1 1_2$ & $-7.96995$ & $0.0232953$        & $0.02300602650371$\\
& $-7.89112$  & $0.0181713$        &\\
$7_1 1_1 1_2$ & $-7.51522$ & $0.00245007$       & $0.002427999643379$\\
& $-7.41983$ & $0.0192443$        &\\
$5_1 3_1 1_2$ & $-6.80829$ & $0.012519$         & $0.012518660407092$\\
& $-6.78357$ & $0.00398018$       &\\
$5_1 1_1 1_2$ & $-6.3398$  &  $0.00135119$      & $0.0013511980854654$\\
& $-6.25276$ & $0.000721952$      &\\
& $-5.86683$ & $0.00112415$       &\\
& $-5.69103$ & $0.000141052$      &\\
& $-5.50802$ & $0.00373666$       &\\
$3_1 1_2$ & $-5.45276$ & $0.000466819$      & $0.0004668223478716$\\
& $-5.06385$ & $0.000380056$      & $0.000166548883431$\\
& $-4.86668$ & $0.0000395855$     &\\
& $-4.64743$ & $0.000014786$      & $0.000016403668846624$\\
& $-4.49059$ & $0.00155894$       &\\
& $-4.25924$ & $0.000109932$      &\\
& $-3.93068$ & $4.12108e-6$       & $4.6283550855711e-6$\\
& $-3.92645$ & $0.0000102752$     &\\
& $-3.77995$ & $0.0000732808$     &\\
& $-3.259$   & $0.0000348879$     &\\
& $-3.12751$ & $0.0000739906$     & $0.00007017294942524$\\
& $-3.04176$ & $1.62543e-6$       &\\
& $-3.00072$ & $1.4684e-6$        & $1.6374806913113e-6$\\
& $-2.32465$ & $2.01172e-8$       &\\
& $-2.26529$ & $7.53377e-6$       &\\
& $-2.02465$ & $5.8028e-7$        & $5.722094498217e-7$\\
& $-1.38078$ & $2.17607e-7$       &\\
& $-1.11438$ & $1.825e-7$         &\\
& $-0.844856$& $1.84715e-9$       & $2.4957509639475e-9$\\
\bottomrule
\end{tabular}
\caption{All Bethe states for $L=12$ with nonzero overlap with the zero-momentum 
 M-G state. The first column show the string content of the eigenstate. The second 
 and third columns show the exact diagonalization resutls for the energy and the 
 overlap, respectively. The last colum is the overlap obtained in the Bethe ansatz 
 approach using the Bethe-Takahashi equations.
}
\label{table:RV:sumruleN12}
\end{table}

\end{document}
























